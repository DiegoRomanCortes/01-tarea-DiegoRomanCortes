% Template:     Informe LaTeX
% Documento:    Archivo principal
% Versión:      7.0.3 (02/08/2020)
% Codificación: UTF-8
%
% Autor: Pablo Pizarro R.
%        Facultad de Ciencias Físicas y Matemáticas
%        Universidad de Chile
%        pablo@ppizarror.com
%
% Manual template: [https://latex.ppizarror.com/informe]
% Licencia MIT:    [https://opensource.org/licenses/MIT]

% CREACIÓN DEL DOCUMENTO
\documentclass[letterpaper,oneside]{article}

% INFORMACIÓN DEL DOCUMENTO
\def\titulodelinforme {Informe Tarea 1}
\def\temaatratar {Probabilidad con distribución $\chi^2$}

\def\autordeldocumento {Diego Román Cortés}
\def\nombredelcurso {Métodos Numéricos para la Ciencia e Ingeniería}
\def\codigodelcurso {FI3104-1}

\def\nombreuniversidad {Universidad de Chile}
\def\nombrefacultad {Facultad de Ciencias Físicas y Matemáticas}
\def\departamentouniversidad {Departamento de Física}
\def\imagendepartamento {departamentos/dfi}
\def\imagendepartamentoescala {0.2}
\def\localizacionuniversidad {Santiago, Chile}

% INTEGRANTES, PROFESORES Y FECHAS
\def\tablaintegrantes {
\begin{tabular}{ll}
	Nombre:
	& \begin{tabular}[t]{l}
		Diego Román
	\end{tabular} \\
	RUT:
	& \begin{tabular}[t]{l}
		20.299.495-4
	\end{tabular} \\
	Profesor:
	& \begin{tabular}[t]{l}
		Valentino González
	\end{tabular} \\
	Auxiliares:
	& \begin{tabular}[t]{l}
		Vicente Donaire \\
		Benjamín Navarrete
	\end{tabular} \\
	\multicolumn{2}{l}{Fecha de entrega: 26 de septiembre} \\
\end{tabular}}{
}

% IMPORTACIÓN DEL TEMPLATE
\input{template}

% INICIO DE PÁGINAS
\begin{document}
	
% PORTADA
\templatePortrait

% CONFIGURACIÓN DE PÁGINA Y ENCABEZADOS
\templatePagecfg

% TABLA DE CONTENIDOS - ÍNDICE
%\templateIndex

% CONFIGURACIONES FINALES
\templateFinalcfg

% ======================= INICIO DEL DOCUMENTO =======================
El problema a resolver es encontrar numéricamente el valor de $a$ para la siguiente ecuación:
$$
0.95 = \P(x<a) =\int_{-\infty}^a pdf(x) dx 
$$
La función $pdf(x)$ es, en este caso, la distribución $\displaystyle \chi^2(x) = \frac{1}{2^{k/2}\Gamma(k/2)}x^{k/2-1}e^{-x/2}$, 
donde $\displaystyle  \Gamma(z) = \int_0^\infty x^{z-1}e^{-x}dx$.

La implementación de la función $\Gamma(z)$ en el código posee el problema del límite de integración superior infinito. Para solucionarlo, se debe realizar un cambio de variables apropiado. Tomando $u = e^{-x}$ ($du = -e^{-x}dx$) la integral queda:
$$
	\int_0^\infty x^{z-1}e^{-x}dx = \int_1^0 -(-\ln(u))^{z-1} du = \int_0^1 \left(\ln(\frac{1}{u})\right)^{z-1}du
$$
La integración numérica de $\Gamma(z)$ requirió del uso del método del punto medio para el caso de la singularidad en $z = 0$, mientras que desde un $\Delta x = 1\cross10^{-7}$ hasta $z=1$ se usó Simpson 1/3.


% FIN DEL DOCUMENTO
\end{document}